%=============================================================================
% File:  month.tex -- Calender showing the month
% Author(s): Juergen Hackl <hackl.j@gmx.at>
% Creation:  27 Jul 2017
% Time-stamp: <Don 2017-07-27 18:51 juergen>
%
% Copyright (c) 2017 Juergen Hackl <hackl.j@gmx.at>
%
% More information on LaTeX: http://www.latex-project.org/
%=============================================================================

%%%%%%%%%%%%%%%%%%%%%%%%%%%%%%%%%%%%%%%%%%%%%%%%%%%%%%%%%%%%%%%%%%%%%%
\documentclass[a4paper,landscape]{article}

%\usepackage[cm]{fullpage}
\usepackage[margin=0pt]{geometry}
 % \geometry{
 % a4paper,
 % total={297mm,210mm},
 % landscape,
 % }



\usepackage{tikz}
\usetikzlibrary{calendar,shadows,patterns}

% Names of Holidays are inserted by employing this macro
\def\termin#1#2{
  \node [anchor=north west, text width= 3.4cm] at
    ($(cal-#1.north west)+(3em, 0em)$) {\tiny{#2}};
}

\usepackage{fontspec}
\setmainfont[ItalicFont={Fira Sans Light Italic},%
BoldFont={Fira Sans},%
BoldItalicFont={Fira Sans Italic}]%
{Fira Sans Light}

%%%%%%%%%%%%%%%%%%%%%%%%%%%%%%%%%%%%%%%%%%%%%%%%%%%%%%%%%%%%%%%%%%%%%%
%% DOCUMENT
\begin{document}

\tikzstyle{every shadow}=[opacity=.4,shadow xshift=2pt,shadow yshift=-2pt,fill=black!60]
\tikzstyle{caption}=[drop shadow,rounded corners=5pt,inner sep=4pt,draw=black!50,fill=black!5]
\tikzstyle{plot legend}=[
   rounded corners=2.5pt,inner xsep=3pt,inner ysep=2pt,
   draw=black!50,fill=white,
   font=\footnotesize,cells={anchor=center},
   nodes={inner sep=2pt,text depth=0.15em,rounded corners=0pt,right}]
 
\begin{tikzpicture}[remember picture,overlay]% anchor = west]% xshift=-0cm]
\draw (current page.north) -- (current page.south);
\end{tikzpicture}

\begin{tikzpicture}[remember picture,overlay,shift={(current page.north west)}]%[remember picture,overlay,xshift=-0cm]

\newcommand{\YYMM}{2017-08}
\newlength{\daywidth}
\newlength{\dayheight}
\newlength{\weekheight}
\newlength{\pagespace}
\setlength{\daywidth}{3.4cm}
\setlength{\dayheight}{2.5cm}
\setlength{\weekheight}{5mm}
\setlength{\pagespace}{12mm}

 
\newcount\monthsheetdate
\newcount\monthsheetweekday
 
\pgfcalendardatetojulian{\YYMM-01}{\monthsheetdate}
\pgfcalendarjuliantoweekday{\the\monthsheetdate}{\monthsheetweekday}
\advance\monthsheetdate by -\the\monthsheetweekday
\pgfcalendarjuliantodate{\the\monthsheetdate}{\myyear}{\mymonth}{\myday}
\edef\monthsheetbegin{\myyear-\mymonth-\myday}
 
\pgfcalendardatetojulian{\YYMM-last}{\monthsheetdate}
\pgfcalendarjuliantoweekday{\the\monthsheetdate}{\monthsheetweekday}
\advance\monthsheetdate by -\the\monthsheetweekday
\advance\monthsheetdate by 6
\pgfcalendarjuliantodate{\the\monthsheetdate}{\myyear}{\mymonth}{\myday}
\edef\monthsheetend{\myyear-\mymonth-\myday}
 
\begin{scope}[shift={(40.5mm,-20mm)}]
\calendar[
name =cal,
    dates=\monthsheetbegin{} to \monthsheetend{},
    execute before day scope={%
      month caption:
      \ifdate{equals=\YYMM-01}{%
        \path (3.5\daywidth,0)
          node[above=1.8cm,
               every month,
               font=\Huge\bf,
               caption,
               inner sep=.5em]
            {\tikzmonthtext{} \tikzyeartext}
        ;
      }{}
      % week day captions:
      \ifdate{at most=\monthsheetbegin+6}{
        \path (\pgfcalendarcurrentweekday\daywidth,0)
          node[xshift=.5\daywidth,
               font=\large\bf,
               above,
               anchor=south,
               draw,
               minimum width=\daywidth,
               minimum height=\weekheight,
               inner sep = 0.0ex,
               outer sep = 0pt,
               text=white,
               week bg
               ] {%
                 \MakeUppercase{\pgfcalendarweekdayname{\pgfcalendarcurrentweekday}}%
          };
      }{}
    },
    execute at begin day scope={%
      \pgftransformxshift{\pgfcalendarcurrentweekday\daywidth}%
    },
    execute after day scope={%
      \ifdate{Wednesday}{%
        \pgftransformxshift{\pagespace}
      }{}%
    },
    execute after day scope={%
      \ifdate{Sunday}{%
        \pgftransformyshift{-\dayheight}%
        \pgftransformxshift{-\pagespace}
      }{}%
    },
%    week list,
%    day xshift=\daywidth,
%    day yshift=\dayheight,
    day bg/.style=,
    every day text/.style={font=\large\bf},
    week bg/.style={fill=black!50},
    day code={%
      \node[anchor=north west,
            draw,
            minimum width=\daywidth,
            minimum height=\dayheight,
            day bg]{};%
      \node[anchor=north west,
            every day text,
            inner sep=7pt]{\tikzdaytext};%
    },
  every month/.append style={anchor=base}
%    month label above centered
  ]
  if (Sunday) [red]
  if (weekend) [day bg/.append style={fill=black!10}]
  if (at most=\YYMM-01+-1,at least=\YYMM-last+1)
    [%day bg/.append style={pattern=north east lines,pattern color=black!50},
     day bg/.append style={fill=black!25},
     every day text/.append style={font=\large}]
;
\end{scope}

\draw (0,0) -- (1,-1);
% \termin{2017-08-02}{Pfingstsonntag}
\end{tikzpicture}


\begin{tikzpicture}[remember picture,overlay]% anchor = west]% xshift=-0cm]
\draw (current page.north east) --++ (-.6,-.6);
\end{tikzpicture}


%%%%%%%%%%%%%%%%%%%%%%%%%%%%%%%%%%%%%%%%%%%%%%%%%%%%%%%%%%%%%%%%%%%%%%

\end{document}

%%% Local Variables:
%%% mode: latex
%%% TeX-master: t
%%% End:
