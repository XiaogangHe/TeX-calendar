%=============================================================================
% File:  month.tex -- Calender showing the month
% Author(s): Juergen Hackl <hackl.j@gmx.at>
% Creation:  27 Jul 2017
% Time-stamp: <Mit 2017-08-02 16:58 juergen>
%
% Copyright (c) 2017 Juergen Hackl <hackl.j@gmx.at>
%
% More information on LaTeX: http://www.latex-project.org/
%=============================================================================

%%%%%%%%%%%%%%%%%%%%%%%%%%%%%%%%%%%%%%%%%%%%%%%%%%%%%%%%%%%%%%%%%%%%%%
\documentclass[a4paper,landscape]{article}

%\usepackage[cm]{fullpage}
\usepackage[margin=0pt]{geometry}
 % \geometry{
 % a4paper,
 % total={297mm,210mm},
 % landscape,
 % }

\usepackage{ifthen}

\def\stripzero#1{\expandafter\stripzerohelp#1}
\def\stripzerohelp#1{\ifx 0#1\expandafter\stripzerohelp\else#1\fi}

\usepackage{tikz}
\usetikzlibrary{calendar,shadows,patterns}
\usetikzlibrary{positioning,calc}
\usetikzlibrary{backgrounds}
% Names of Holidays are inserted by employing this macro
\def\termin#1#2{
  \node [anchor=north west, text width= 3.4cm] at
    ($(cal-#1.north west)+(3em, 0em)$) {\tiny{#2}};
}

\usepackage{fontspec}
\setmainfont[ItalicFont={Fira Sans Light Italic},%
BoldFont={Fira Sans},%
BoldItalicFont={Fira Sans Italic}]%
{Fira Sans Light}


% \def\termin#1#2{
% \node [text width= 3.4cm] at
% ($(cal-#1)+(0em, 0em)$) {\textcolor{red\tiny{#2}};
%  }


%%%%%%%%%%%%%%%%%%%%%%%%%%%%%%%%%%%%%%%%%%%%%%%%%%%%%%%%%%%%%%%%%%%%%%
%% DOCUMENT
\begin{document}

\tikzstyle{every shadow}=[opacity=.4,shadow xshift=2pt,shadow yshift=-2pt,fill=black!60]
\tikzstyle{caption}=[drop shadow,rounded corners=5pt,inner sep=4pt,draw=black!50,fill=black!5]
\tikzstyle{plot legend}=[
   rounded corners=2.5pt,inner xsep=3pt,inner ysep=2pt,
   draw=black!50,fill=white,
   font=\footnotesize,cells={anchor=center},
   nodes={inner sep=2pt,text depth=0.15em,rounded corners=0pt,right}]

\begin{tikzpicture}[remember picture,overlay,shift={(current page.north west)}]%[remember picture,overlay,xshift=-0cm]

\newcommand{\YYMM}{2017-08}
\newcommand{\YY}{2017}
\newlength{\daywidth}
\newlength{\dayheight}
\newlength{\weekheight}
\newlength{\pagespace}
\newlength{\boxspace}
\newlength{\bugfixspace}
\newcommand{\notesoffset}{0}

\setlength{\daywidth}{9mm}
\setlength{\dayheight}{15.5mm}
\newcommand{\dayh}{15}
\setlength{\weekheight}{5mm}
\newcommand{\weekh}{5}
\setlength{\pagespace}{12mm}
\setlength{\boxspace}{2mm}
\setlength{\bugfixspace}{0mm}
%\setlength{\notesoffset}{0mm}
\newcommand{\notesheight}{4.5}

\newlength{\defaultpgflinewidth}
\setlength{\defaultpgflinewidth}{.5pt}%{\pgflinewidth}

\definecolor{linecolor}{gray}{.22}
\definecolor{headingcolor}{gray}{.22}
\definecolor{weekcolor}{gray}{0.35}
\definecolor{weekendcolor}{gray}{0.8}
\definecolor{monthcolor}{gray}{0.6}
\definecolor{htextcolor}{gray}{1}
\definecolor{textcolor}{gray}{.22}
\definecolor{termincolor}{RGB}{58,110,143}
\definecolor{sundaycolor}{gray}{.22}
\definecolor{indexcolor}{gray}{.22}

% \definecolor{linecolor}{RGB}{30,30,30}
% \definecolor{headingcolor}{RGB}{30,30,30}
% \definecolor{weekcolor}{gray}{0.5}
% \definecolor{weekendcolor}{gray}{0.9}
% \definecolor{monthcolor}{gray}{0.75}
% \definecolor{htextcolor}{gray}{1}
% \definecolor{textcolor}{RGB}{30,30,30}
% \definecolor{sundaycolor}{RGB}{58,110,143}

\newcount\monthsheetdate
\newcount\monthsheetweekday
\newcount\weekrows

\pgfcalendardatetojulian{\YYMM-01}{\monthsheetdate}
\pgfcalendarjuliantoweekday{\the\monthsheetdate}{\monthsheetweekday}
\advance\monthsheetdate by -\the\monthsheetweekday
\pgfcalendarjuliantodate{\the\monthsheetdate}{\myyear}{\mymonth}{\myday}
\edef\monthsheetbegin{\myyear-\mymonth-\myday}
 
\pgfcalendardatetojulian{\YYMM-last}{\monthsheetdate}
\pgfcalendarjuliantoweekday{\the\monthsheetdate}{\monthsheetweekday}
\advance\monthsheetdate by -\the\monthsheetweekday
\advance\monthsheetdate by 6
\pgfcalendarjuliantodate{\the\monthsheetdate}{\myyear}{\mymonth}{\myday}
\edef\monthsheetend{\myyear-\mymonth-\myday}
 
\tikzset{myline/.style={draw, linecolor, line width = \defaultpgflinewidth}}

 \tikzstyle{monthnode}=[draw,color=indexcolor,
        fill=indexcolor, minimum width=\dayheight,
                  minimum height=6mm,
                  anchor = south east,
                  rotate = 90,
                  text = htextcolor,]

\begin{scope}[shift={(13.4mm,-13mm)}]
\draw[myline,fill=monthcolor] (0,0) rectangle (31\daywidth,-12\dayheight);
%\newcount\mycount
\calendar(cal)[
    dates=\YYMM-01 to \YYMM-last,
    execute before day scope={%
      month caption:
      \ifdate{equals=\YYMM-01}{%
        \newcount\myBcount
%
        \pgfcalendardatetojulian{\YYMM-01}{\myBcount}
        \pgfcalendarjuliantodate{\the\myBcount}{\myByear}{\myBmonth}{\myBday}
       \node at (0,0) [monthnode]{\MakeUppercase{\pgfcalendarmonthshortname{\myBmonth} \myBmonth}};
      }{}
      % week day captions:
      \ifdate{at most=\YYMM-last}{
        \path (\pgfcalendarcurrentday\daywidth-\daywidth,0)
          node[xshift=.5\daywidth+\defaultpgflinewidth/2,
          node distance=-\defaultpgflinewidth,
               font=\normalsize,
               above,
               anchor=south,
               myline,
               minimum width=\daywidth,
               minimum height=\weekheight,
               inner sep = 0.0pt,
               outer sep = 0pt,
               text=htextcolor,
               week bg
               ] {%
                 \MakeUppercase{\pgfcalendarcurrentday}%
          };
      }{}
    },
    execute at begin day scope={%
      \pgftransformxshift{\pgfcalendarcurrentday\daywidth-\daywidth}
    },
    day bg/.style=,
    every day text/.style={font=\footnotesize\bf},
    week bg/.style={fill=weekcolor},
    day code={%
      \node[anchor=north west,
            myline,
            minimum width=\daywidth,
            minimum height=\dayheight,
            fill=white,day bg,]{};%
      \node[anchor=north west,
            text width=\daywidth,
            minimum height=5mm,
            every day text,align=center,
            inner sep=0pt,]{\pgfcalendarweekdayshortname{\pgfcalendarcurrentweekday}};%
    },
  every month/.append style={anchor=base},
  ]
  if (weekend) [day bg/.append style={fill=weekendcolor}];



\begin{scope}[shift={(0mm,-\dayheight)}]
\renewcommand{\YYMM}{2017-09}
\calendar(cal)[dates=\YYMM-01 to \YYMM-last, execute before day scope={month caption:\ifdate{equals=\YYMM-01}{\newcount\myBcount \pgfcalendardatetojulian{\YYMM-01}{\myBcount} \pgfcalendarjuliantodate{\the\myBcount}{\myByear}{\myBmonth}{\myBday} \node at (0,0) [monthnode]{\MakeUppercase{\pgfcalendarmonthshortname{\myBmonth} \myBmonth}}; }{}}, execute at begin day scope={\pgftransformxshift{\pgfcalendarcurrentday\daywidth-\daywidth}}, day bg/.style=,every day text/.style={font=\footnotesize\bf}, week bg/.style={fill=weekcolor}, day code={\node[anchor=north west,myline,minimum width=\daywidth,minimum height=\dayheight,fill=white,day bg]{};\node[anchor=north west,text width=\daywidth,minimum height=5mm,every day text,align=center, inner sep=0pt]{\pgfcalendarweekdayshortname{\pgfcalendarcurrentweekday}};},every month/.append style={anchor=base},]if (weekend) [day bg/.append style={fill=weekendcolor}];
\end{scope}

\begin{scope}[shift={(0mm,-2\dayheight)}]
\renewcommand{\YYMM}{2017-10}
\calendar(cal)[dates=\YYMM-01 to \YYMM-last, execute before day scope={month caption:\ifdate{equals=\YYMM-01}{\newcount\myBcount \pgfcalendardatetojulian{\YYMM-01}{\myBcount} \pgfcalendarjuliantodate{\the\myBcount}{\myByear}{\myBmonth}{\myBday} \node at (0,0) [monthnode]{\MakeUppercase{\pgfcalendarmonthshortname{\myBmonth} \myBmonth}}; }{}}, execute at begin day scope={\pgftransformxshift{\pgfcalendarcurrentday\daywidth-\daywidth}}, day bg/.style=,every day text/.style={font=\footnotesize\bf}, week bg/.style={fill=weekcolor}, day code={\node[anchor=north west,myline,minimum width=\daywidth,minimum height=\dayheight,fill=white,day bg]{};\node[anchor=north west,text width=\daywidth,minimum height=5mm,every day text,align=center, inner sep=0pt]{\pgfcalendarweekdayshortname{\pgfcalendarcurrentweekday}};},every month/.append style={anchor=base},]if (weekend) [day bg/.append style={fill=weekendcolor}];
\end{scope}

\begin{scope}[shift={(0mm,-3\dayheight)}]
\renewcommand{\YYMM}{2017-11}
\calendar(cal)[dates=\YYMM-01 to \YYMM-last, execute before day scope={month caption:\ifdate{equals=\YYMM-01}{\newcount\myBcount \pgfcalendardatetojulian{\YYMM-01}{\myBcount} \pgfcalendarjuliantodate{\the\myBcount}{\myByear}{\myBmonth}{\myBday} \node at (0,0) [monthnode]{\MakeUppercase{\pgfcalendarmonthshortname{\myBmonth} \myBmonth}}; }{}}, execute at begin day scope={\pgftransformxshift{\pgfcalendarcurrentday\daywidth-\daywidth}}, day bg/.style=,every day text/.style={font=\footnotesize\bf}, week bg/.style={fill=weekcolor}, day code={\node[anchor=north west,myline,minimum width=\daywidth,minimum height=\dayheight,fill=white,day bg]{};\node[anchor=north west,text width=\daywidth,minimum height=5mm,every day text,align=center, inner sep=0pt]{\pgfcalendarweekdayshortname{\pgfcalendarcurrentweekday}};},every month/.append style={anchor=base},]if (weekend) [day bg/.append style={fill=weekendcolor}];
\end{scope}

\begin{scope}[shift={(0mm,-4\dayheight)}]
\renewcommand{\YYMM}{2017-12}
\calendar(cal)[dates=\YYMM-01 to \YYMM-last, execute before day scope={month caption:\ifdate{equals=\YYMM-01}{\newcount\myBcount \pgfcalendardatetojulian{\YYMM-01}{\myBcount} \pgfcalendarjuliantodate{\the\myBcount}{\myByear}{\myBmonth}{\myBday} \node at (0,0) [monthnode]{\MakeUppercase{\pgfcalendarmonthshortname{\myBmonth} \myBmonth}}; }{}}, execute at begin day scope={\pgftransformxshift{\pgfcalendarcurrentday\daywidth-\daywidth}}, day bg/.style=,every day text/.style={font=\footnotesize\bf}, week bg/.style={fill=weekcolor}, day code={\node[anchor=north west,myline,minimum width=\daywidth,minimum height=\dayheight,fill=white,day bg]{};\node[anchor=north west,text width=\daywidth,minimum height=5mm,every day text,align=center, inner sep=0pt]{\pgfcalendarweekdayshortname{\pgfcalendarcurrentweekday}};},every month/.append style={anchor=base},]if (weekend) [day bg/.append style={fill=weekendcolor}];
\end{scope}


\begin{scope}[shift={(0mm,-5\dayheight)}]
\renewcommand{\YYMM}{2018-01}
\calendar(cal)[dates=\YYMM-01 to \YYMM-last, execute before day scope={month caption:\ifdate{equals=\YYMM-01}{\newcount\myBcount \pgfcalendardatetojulian{\YYMM-01}{\myBcount} \pgfcalendarjuliantodate{\the\myBcount}{\myByear}{\myBmonth}{\myBday} \node at (0,0) [monthnode]{\MakeUppercase{\pgfcalendarmonthshortname{\myBmonth} \myBmonth}}; }{}}, execute at begin day scope={\pgftransformxshift{\pgfcalendarcurrentday\daywidth-\daywidth}}, day bg/.style=,every day text/.style={font=\footnotesize\bf}, week bg/.style={fill=weekcolor}, day code={\node[anchor=north west,myline,minimum width=\daywidth,minimum height=\dayheight,fill=white,day bg]{};\node[anchor=north west,text width=\daywidth,minimum height=5mm,every day text,align=center, inner sep=0pt]{\pgfcalendarweekdayshortname{\pgfcalendarcurrentweekday}};},every month/.append style={anchor=base},]if (weekend) [day bg/.append style={fill=weekendcolor}];
\end{scope}


\begin{scope}[shift={(0mm,-6\dayheight)}]
\renewcommand{\YYMM}{2018-02}
\calendar(cal)[dates=\YYMM-01 to \YYMM-last, execute before day scope={month caption:\ifdate{equals=\YYMM-01}{\newcount\myBcount \pgfcalendardatetojulian{\YYMM-01}{\myBcount} \pgfcalendarjuliantodate{\the\myBcount}{\myByear}{\myBmonth}{\myBday} \node at (0,0) [monthnode]{\MakeUppercase{\pgfcalendarmonthshortname{\myBmonth} \myBmonth}}; }{}}, execute at begin day scope={\pgftransformxshift{\pgfcalendarcurrentday\daywidth-\daywidth}}, day bg/.style=,every day text/.style={font=\footnotesize\bf}, week bg/.style={fill=weekcolor}, day code={\node[anchor=north west,myline,minimum width=\daywidth,minimum height=\dayheight,fill=white,day bg]{};\node[anchor=north west,text width=\daywidth,minimum height=5mm,every day text,align=center, inner sep=0pt]{\pgfcalendarweekdayshortname{\pgfcalendarcurrentweekday}};},every month/.append style={anchor=base},]if (weekend) [day bg/.append style={fill=weekendcolor}];
\end{scope}


\begin{scope}[shift={(0mm,-7\dayheight)}]
\renewcommand{\YYMM}{2018-03}
\calendar(cal)[dates=\YYMM-01 to \YYMM-last, execute before day scope={month caption:\ifdate{equals=\YYMM-01}{\newcount\myBcount \pgfcalendardatetojulian{\YYMM-01}{\myBcount} \pgfcalendarjuliantodate{\the\myBcount}{\myByear}{\myBmonth}{\myBday} \node at (0,0) [monthnode]{\MakeUppercase{\pgfcalendarmonthshortname{\myBmonth} \myBmonth}}; }{}}, execute at begin day scope={\pgftransformxshift{\pgfcalendarcurrentday\daywidth-\daywidth}}, day bg/.style=,every day text/.style={font=\footnotesize\bf}, week bg/.style={fill=weekcolor}, day code={\node[anchor=north west,myline,minimum width=\daywidth,minimum height=\dayheight,fill=white,day bg]{};\node[anchor=north west,text width=\daywidth,minimum height=5mm,every day text,align=center, inner sep=0pt]{\pgfcalendarweekdayshortname{\pgfcalendarcurrentweekday}};},every month/.append style={anchor=base},]if (weekend) [day bg/.append style={fill=weekendcolor}];
\end{scope}

\begin{scope}[shift={(0mm,-8\dayheight)}]
\renewcommand{\YYMM}{2018-04}
\calendar(cal)[dates=\YYMM-01 to \YYMM-last, execute before day scope={month caption:\ifdate{equals=\YYMM-01}{\newcount\myBcount \pgfcalendardatetojulian{\YYMM-01}{\myBcount} \pgfcalendarjuliantodate{\the\myBcount}{\myByear}{\myBmonth}{\myBday} \node at (0,0) [monthnode]{\MakeUppercase{\pgfcalendarmonthshortname{\myBmonth} \myBmonth}}; }{}}, execute at begin day scope={\pgftransformxshift{\pgfcalendarcurrentday\daywidth-\daywidth}}, day bg/.style=,every day text/.style={font=\footnotesize\bf}, week bg/.style={fill=weekcolor}, day code={\node[anchor=north west,myline,minimum width=\daywidth,minimum height=\dayheight,fill=white,day bg]{};\node[anchor=north west,text width=\daywidth,minimum height=5mm,every day text,align=center, inner sep=0pt]{\pgfcalendarweekdayshortname{\pgfcalendarcurrentweekday}};},every month/.append style={anchor=base},]if (weekend) [day bg/.append style={fill=weekendcolor}];
\end{scope}

\begin{scope}[shift={(0mm,-9\dayheight)}]
\renewcommand{\YYMM}{2018-05}
\calendar(cal)[dates=\YYMM-01 to \YYMM-last, execute before day scope={month caption:\ifdate{equals=\YYMM-01}{\newcount\myBcount \pgfcalendardatetojulian{\YYMM-01}{\myBcount} \pgfcalendarjuliantodate{\the\myBcount}{\myByear}{\myBmonth}{\myBday} \node at (0,0) [monthnode]{\MakeUppercase{\pgfcalendarmonthshortname{\myBmonth} \myBmonth}}; }{}}, execute at begin day scope={\pgftransformxshift{\pgfcalendarcurrentday\daywidth-\daywidth}}, day bg/.style=,every day text/.style={font=\footnotesize\bf}, week bg/.style={fill=weekcolor}, day code={\node[anchor=north west,myline,minimum width=\daywidth,minimum height=\dayheight,fill=white,day bg]{};\node[anchor=north west,text width=\daywidth,minimum height=5mm,every day text,align=center, inner sep=0pt]{\pgfcalendarweekdayshortname{\pgfcalendarcurrentweekday}};},every month/.append style={anchor=base},]if (weekend) [day bg/.append style={fill=weekendcolor}];
\end{scope}

\begin{scope}[shift={(0mm,-10\dayheight)}]
\renewcommand{\YYMM}{2018-06}
\calendar(cal)[dates=\YYMM-01 to \YYMM-last, execute before day scope={month caption:\ifdate{equals=\YYMM-01}{\newcount\myBcount \pgfcalendardatetojulian{\YYMM-01}{\myBcount} \pgfcalendarjuliantodate{\the\myBcount}{\myByear}{\myBmonth}{\myBday} \node at (0,0) [monthnode]{\MakeUppercase{\pgfcalendarmonthshortname{\myBmonth} \myBmonth}}; }{}}, execute at begin day scope={\pgftransformxshift{\pgfcalendarcurrentday\daywidth-\daywidth}}, day bg/.style=,every day text/.style={font=\footnotesize\bf}, week bg/.style={fill=weekcolor}, day code={\node[anchor=north west,myline,minimum width=\daywidth,minimum height=\dayheight,fill=white,day bg]{};\node[anchor=north west,text width=\daywidth,minimum height=5mm,every day text,align=center, inner sep=0pt]{\pgfcalendarweekdayshortname{\pgfcalendarcurrentweekday}};},every month/.append style={anchor=base},]if (weekend) [day bg/.append style={fill=weekendcolor}];
\end{scope}

\begin{scope}[shift={(0mm,-11\dayheight)}]
\renewcommand{\YYMM}{2018-07}
\calendar(cal)[dates=\YYMM-01 to \YYMM-last, execute before day scope={month caption:\ifdate{equals=\YYMM-01}{\newcount\myBcount \pgfcalendardatetojulian{\YYMM-01}{\myBcount} \pgfcalendarjuliantodate{\the\myBcount}{\myByear}{\myBmonth}{\myBday} \node at (0,0) [monthnode]{\MakeUppercase{\pgfcalendarmonthshortname{\myBmonth} \myBmonth}}; }{}}, execute at begin day scope={\pgftransformxshift{\pgfcalendarcurrentday\daywidth-\daywidth}}, day bg/.style=,every day text/.style={font=\footnotesize\bf}, week bg/.style={fill=weekcolor}, day code={\node[anchor=north west,myline,minimum width=\daywidth,minimum height=\dayheight,fill=white,day bg]{};\node[anchor=north west,text width=\daywidth,minimum height=5mm,every day text,align=center, inner sep=0pt]{\pgfcalendarweekdayshortname{\pgfcalendarcurrentweekday}};},every month/.append style={anchor=base},]if (weekend) [day bg/.append style={fill=weekendcolor}];
\end{scope}


\end{scope}



% \draw (0,-21) --++ (0.6,.8);
% \draw (0,0) -- (.6,-.8);
% \termin{2017-08-02}{Pfingstsonntag}
\end{tikzpicture}

\begin{tikzpicture}[remember picture,overlay]% anchor = west]% xshift=-0cm]
\tikzset{mymarkline/.style={draw, black!50, line width = 0.6pt}}
\tikzset{mycircle/.style={black!50}}

\newlength{\marklength}
\setlength{\marklength}{1.5mm}

\newlength{\diam}
\setlength{\diam}{1pt}

% \fill [mycircle] ($(current page.north)-(0,.8)$) circle (\diam);
% \fill [mycircle] ($(current page.north)-(0,3.4)$) circle (\diam);
% \fill [mycircle] ($(current page.north)-(0,6.4)$) circle (\diam);
% \fill [mycircle] ($(current page.north)-(0,9)$) circle (\diam);
\fill [mycircle] ($(current page.south)+(0,.8)$) circle (\diam);
% \fill [mycircle] ($(current page.south)+(0,3.4)$) circle (\diam);
% \fill [mycircle] ($(current page.south)+(0,6.4)$) circle (\diam);
% \fill [mycircle] ($(current page.south)+(0,9)$) circle (\diam);


\draw [mymarkline] (current page.north west) --++ (\marklength,0);
\draw [mymarkline] (current page.north west) --++ (0,-\marklength);

\draw [mymarkline] (current page.north east) --++ (-\marklength,0);
\draw [mymarkline] (current page.north east) --++ (0,-\marklength);


\draw [mymarkline] (current page.south west) --++ (\marklength,0);
\draw [mymarkline] (current page.south west) --++ (0,\marklength);

\draw [mymarkline] (current page.south east) --++ (-\marklength,0);
\draw [mymarkline] (current page.south east) --++ (0,\marklength);


\end{tikzpicture}



%%%%%%%%%%%%%%%%%%%%%%%%%%%%%%%%%%%%%%%%%%%%%%%%%%%%%%%%%%%%%%%%%%%%%%

\end{document}

%%% Local Variables:
%%% mode: latex
%%% TeX-master: t
%%% End:
